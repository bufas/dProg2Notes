\subsection{Dispositionsforslag 2}

\begin{itemize}
    \item \textbf{Nedarvning}
    \begin{itemize}
        \item “arve” metoder fra superklassen
        \item visibility modifiers
        \item \textbf(super keyword)
        \item Kan kun nedarve fra én klasse
        \item Regler ved nedarvning
        \begin{itemize}
            \item "is-a relationen" (\textless subtype\textgreater is a \textless supertype\textgreater)
            \item Liskov substitution\footnote{Husk at Barbara Liskov er en kvinde} (en subklasse kan altid indsættes i stedet for en superklasse)
            \item Eksempel der overholder Liskov substitution
            \begin{itemize}
                \item Både is-a og Liskov glæder (Crab extends Animal (fra dIntProg) eller JonglerendeKlovn extends Klovn)
                \item Kun is-a gælder (Square extends Rectangle\footnote{Da square er mere restriktiv end rectangle})
            \end{itemize}
            
            \item Preconditions (maks lige så stærke)
            \item Postconditions (mindst lige så stærke)
        \end{itemize}
        
        \item Centralisering af kode ved at flytte det op i supertype (referer til aflevering om BankAccount)
    \end{itemize}
    
    \item \textbf{Polymorfi}
    \begin{itemize}
        \item Flere metoder med samme navn
        \begin{itemize}
            \item typesystemet sørger for korrekt kald
            \item dynamisk/statisk type
        \end{itemize}
        
        \item Eksempel på polymorfi
        \begin{itemize}
            \item Artist a = new Klovn();
        \end{itemize}
    \end{itemize}

    \item \textbf{Abstrakte klasser}
    \begin{itemize}
        \item Flere klasser har ens metoder
        \item Centralisering af kode
        \item tvinge subklasser til at implementere metoder
    \end{itemize}
    
    \item \textbf{TEMPLATE METHOD pattern}
    \begin{itemize}
        \item Eks. Gå over vejen (menneske og hund)
        \begin{itemize}
            \item primitive metoder (kig() og gå())
            \item Da hunden har 4 ben går den anderledes end mennesket
            \item Den kigger sig nok heller ikke så godt for
        \end{itemize}
    \end{itemize}
\end{itemize}

