\subsection{Dispositionsforslag 2}

\begin{itemize}
    \item hvad er exceptions og hvad bruges de til
    \item hvad ville man gøre uden exceptions\footnote{I sproget C returnerer de fleste metoder en værdi der fortæller om kaldet gik godt eller skidt. For systemkald repræsenterer en negativ returværdi oftest at noget er gået galt.} (sproget C har ingen exceptions)
    \item \textbf{2 typer exceptions} (checked og unchecked)
    \begin{itemize}
        \item Tegn exceptions klassehierarki (Throwable, Exception, RuntimeException, Error)
        \item Egne exceptions (Hvilken klasse skal nedarves fra)
        \item Try / catch / finally (husk finally køres ALTID, og bruges ofte til cleanup)
        \item Throws keyword
    \end{itemize}
    
    \item \textbf{Fil håndtering}
    \begin{itemize}
        \item Meget kan gå galt
        \item Eksempler (Filen eksisterer ikke, ingen adgang til filen)
        \item Relater evt. til aflevering om filhåndtering
    \end{itemize}
    
    \item \textbf{Defensive programming}
    \begin{itemize}
        \item Minimerer antallet af fejl
        \item Tager højde for alt (især brugerinput)
        \item Brugeren er ond
        \item Programmør fejl
        \item Sørg for at programmet ikke crasher
    \end{itemize}
\end{itemize}
