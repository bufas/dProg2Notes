\subsection{Dispositionsforslag 2}

Det vigtige i dette emne er at du vælger så få at du har tid til at gå i dybden med dem alle, men ikke det er også vigtigt at du kan fylde tiden ud. Hellere have forberedt et pattern for meget end et for lidt. \\
For hvert pattern skal der være styr på UML diagrammet og hvilket problem det løser. Derudover ville det også være fint at kunne give et eksempel på brug.

\begin{itemize}
    \item \textbf{Composite}
    \begin{itemize}
        \item mønstret realiserer at en samling af objekter kan håndteres som et enkelt objekt. (Eks. en dagligvare og en indkøbspose med varer)
    \end{itemize}
    
    \item \textbf{Adapter}
    \begin{itemize}
        \item et objekt skal bruges af en klasse uden at der ændres på objektet. Objektet implementerer ikke det rigtige interface og skal adaptes til det rigtige interface.
        \item En adapter-klasse oprettes som implementerer det rigtige interface og bruger objektets metoder til at realisere de nye metoder
    \end{itemize}
    
    \item \textbf{Template Method}
    \begin{itemize}
        \item Relater til frameworks og nævn prototype pattern
        \item Hvad: en klasse der kalder primitive metoder på subklasser af et interface og kombinerer disse til en kompleks metode.
        \item Hvornår: en algoritme bruges i flere klasser. Algoritmen kan brydes ned til mindre dele. Delalgoritmerne defineres i klasserne. En superklasse kalder de primitive algoritmer i en rækkefølge så det danner den komplekse algoritme. Algoritmen er nu samlet et sted.
    \end{itemize}
    
    \item \textbf{Proxy}
    \begin{itemize}
        \item Hvad: en “stand-in” klasse for det oprindelige subjekt. Har samme metoder, dog med nogle modifikationer der gør den fordelagtig.
        \item Hvordan: en proxyklasse implementerer samme interface som det oprindelige subjekt. Proxyen kalder metoderne på subjektet.
    \end{itemize}
    
\end{itemize}
