\subsection{Dispositionsforslag 2}

\begin{itemize}
    \item \textbf{Tråde}
    \begin{itemize}
        \item Hvad er en Tråd
        \begin{itemize}
            \item Producer / Consumer\footnote{En producer tråd laver data, fx læser ord i en fil. Oftest tilføjes denne data til en kø. Consumer tråd bruger denne data, fx lægger antal ord sammen for flere filer. Normalvis venter consumertråden på at der kommer data i køen.}
        \end{itemize}
        
        \item Hvorfor bruge threads
        \begin{itemize}
            \item Referer til binomial aflevering, hvor beregningen kunne køres i en separat tråd, så GUIen ikke ville fryse
            \item Kan benytte alle processorens kerner
        \end{itemize}
        
        \item \textbf{Runnable} - Hvordan laves en tråd
        \begin{itemize}
            \item Implementer Runnable interfacet og benyt start() metoden på Thread
            \item 4 states (new, runnable, blocked, dead) - lav tegning
        \end{itemize}
        
        \item Interrupts - try/catch i run() metoden
    \end{itemize}
    
    \item \textbf{Synkronisering}
    \begin{itemize}
        \item \textbf{Race conditions}
        \item \textbf{Locking}
        \begin{itemize}
            \item Eksplicitte locks fra java.util.concurrency (\verb|await()| / \verb|signalAll()|)
            \item Synchronized keyword (Nemmere at bruge, men mindre fleksible)
            \item \textbf{Deadlocks og livelocks\footnote{Livelocks er ikke ligeså vigtive som deadlocks, så undlad at snakke om disse hvis du ikke har styr på dem.}}
        \end{itemize}
        
        \item Køer
        \begin{itemize}
            \item Udveksling af data mellem tråde
            \item Benyttes ofte i forbindesle med producer/consumer
            \item Undgå raceconditions ved at benytte en synkroniseret kø fx LinkedBlockingQueue
        \end{itemize}
    \end{itemize}
\end{itemize}

