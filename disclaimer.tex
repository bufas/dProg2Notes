\subsection{Læs dette først!}

Dokumentet er ment som en hjælp til at læse op til Programmering 2 eksamen. Det er på ingen måde komplet, og vi giver ingen garanti for korrektheden af indholdet. Hvis du opdager fejl, eller har forslag til forbedringer, er du mere end velkommen til at skrive til en af os. Emails findes på forsiden af dokumentet.

Dokumentet er opbygget på den måde at det er et kapitel for hvert eksamensemne. Først er der noter til emnet og til sidst to eksempler på dispositioner. Nogle af punkterne i dispositionerne er markeret med fed skrift - det er disse punkter, vi anbefaler, at du skriver på tavlen, efter du har trukket dit emne.

\subsection{Generelt om mundtlig eksamen}
Det er vigtigt at forberede sig godt på emnerne og komme i gang i god tid. Det tager som regel længere tid at forberede sig, end man regner med. En god teknik er at arbejde sammen i grupper af to eller tre personer og fremlægge for hinanden. På denne måde får man feedback med det samme og kan således nå at forbedre sin fremlæggelse inden eksamen. Tag også tid på dine fremlæggelser. Det er vigtigt at tiden passer nogenlunde, så du får sagt alt det, du gerne vil, og at du ikke er færdig efter fem minutter. "Øvelse, øvelse, øvelse... og atter øvelse" - Holger A. Nielsen.

Det kan derudover være en god ide at overvære hinandens eksamener. Aftal med din læsemakker, at I går med ind til hinandens eksamener. På den måde kan I få ekstra feedback på jeres præsentation, og dermed forbedre jer til næste eksamen. Husk på, at skulderklap ikke hjælper ret meget, så man skal kunne tåle at høre, hvad man har gjort galt. Udover at få god feedback finder du også ud af, hvilke spørgsmål censor og eksaminator stiller, og muligvis hvilke emner, der bør undgås.

\subsection{Fremlæggelse af patterns}
Når du skal fremlægge et pattern er det vigtigt at have styr på tre ting:
\begin{itemize}
    \item UML diagrammet for det pågældende pattern skal være helt på plads. Pilene skal tegnes korrekt og der skal være styr på relationerne mellem klasser og interfaces. Det kan være en god ide at tegne dette som noget af det første, da du resten af tiden vil kunne referere til det og kigge på det, hvis du glemmer noget.
    \item Hilvket problem dette pattern løser og i hvilke situationer det er relevant at benytte. Dette er skrevet på punktform for hvert pattern i Horstmann-bogen, så det burde ikke være så svært at finde ud af.
    \item Et konkret ksempel på brug. Sørg for at have et godt eksempel som du føler dig tryg ved (brug fx afleveringerne, hvis det specifikke pattern er blevet anvendt i en aflevering). Hvis du ikke selv bringer det op kan du risikere at blive spurgt. Det kan være svært at finde på et godt eksempel midt under eksaminationen.
\end{itemize}
